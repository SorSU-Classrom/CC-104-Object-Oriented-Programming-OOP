%%%%%%%%%%%%%%%%%%%%%%%%%%%%%%%%%%%%%%%%%%%%%%%%%%%
%% LaTeX book template                           %%
%% Author:  Amber Jain (http://amberj.devio.us/) %%
%% License: ISC license                          %%
%%%%%%%%%%%%%%%%%%%%%%%%%%%%%%%%%%%%%%%%%%%%%%%%%%%

\documentclass[a4paper,11pt,oneside]{book}
\usepackage{../../modulestyle}

%%%%%%%%%%%%%%%%%%%%%%%%%%%%%%%%%%%%%%%%%%%%%%%%%%%%%%%%%
% Source: http://en.wikibooks.org/wiki/LaTeX/Hyperlinks %
%%%%%%%%%%%%%%%%%%%%%%%%%%%%%%%%%%%%%%%%%%%%%%%%%%%%%%%%%

%%%%%%%%%%%%%%%%%%%%%%%%%%%%%%%%%%%%%%%%%%%%%%%%%%%%%%%%%%%%%%%%%%%%%%%%%%%%%%%%
% 'dedication' environment: To add a dedication paragraph at the start of book %
% Source: http://www.tug.org/pipermail/texhax/2010-June/015184.html            %
%%%%%%%%%%%%%%%%%%%%%%%%%%%%%%%%%%%%%%%%%%%%%%%%%%%%%%%%%%%%%%%%%%%%%%%%%%%%%%%%
\newenvironment{dedication}
{
   \cleardoublepage
   \thispagestyle{empty}
   \vspace*{\stretch{1}}
   \hfill\begin{minipage}[t]{0.66\textwidth}
   \raggedright
}
{
   \end{minipage}
   \vspace*{\stretch{3}}
   \clearpage
}

%%%%%%%%%%%%%%%%%%%%%%%%%%%%%%%%%%%%%%%%%%%%%%%%
% Chapter quote at the start of chapter        %
% Source: http://tex.stackexchange.com/a/53380 %
%%%%%%%%%%%%%%%%%%%%%%%%%%%%%%%%%%%%%%%%%%%%%%%%
\makeatletter
\renewcommand{\@chapapp}{}% Not necessary...
\newenvironment{chapquote}[2][2em]
  {\setlength{\@tempdima}{#1}%
   \def\chapquote@author{#2}%
   \parshape 1 \@tempdima \dimexpr\textwidth-2\@tempdima\relax%
   \itshape}
  {\par\normalfont\hfill--\ \chapquote@author\hspace*{\@tempdima}\par\bigskip}
\makeatother

%%%%%%%%%%%%%%%%%%%%%%%%%%%%%%%%%%%%%%%%%%%%%%%%%%%
% First page of book which contains 'stuff' like: %
%  - Book title, subtitle                         %
%  - Book author name                             %
%%%%%%%%%%%%%%%%%%%%%%%%%%%%%%%%%%%%%%%%%%%%%%%%%%%

\newcommand{\CourseTitle}{Object-Oriented Programming}
\newcommand{\ChapterNumber}{3}
\newcommand{\ChapterTitle}{Methods}
\newcommand{\CodingNumber}{3.10}
\newcommand{\CodingTitle}{A little bit of Physics}
\newcommand{\SubmissionDeadline}{October 18, 2024}
\newcommand{\SubmissionDeadlineText}{on or before \SubmissionDeadline}
\newcommand{\SubmissionTemplateURL}{https://docs.google.com/document/d/1sctvVLgpPSVnXN82k6LsOPSPApKp2rV0/edit?usp=drive_link&ouid=112709378145681657270&rtpof=true&sd=true}

\newcommand{\BookTitle}{Coding Exercise: \CodingNumber - \CodingTitle}
\newcommand{\BookTitleFootnote}{A coding exercise for
Chapter \ChapterNumber of the Study Guide on the course \CourseTitle.}

\newcommand{\BookSubtitle}{Chapter \ChapterNumber: \ChapterTitle}
\newcommand{\BookSubtitleFootnote}{This chapter covers the basics of methods in Java.}

\newcommand{\BookAuthorFirstName}{Jarrian Vince}
\newcommand{\BookAuthorLastName}{Gojar}
\newcommand{\BookAuthorName}{Jarrian Vince G. Gojar}
\newcommand{\BookAuthorURL}{https://github.com/godkingjay}

\newcommand{\GoogleDriveURLBSCSTwoOne}{https://drive.google.com/drive/folders/1c56xFCJgFh6FWQQ4iZ-UuKKcWioF8pgs?usp=sharing}
\newcommand{\GoogleDriveURLBSCSTwoTwo}{https://drive.google.com/drive/folders/1jANc3o6atOYbHyoJZ6b-j-nDlTknEiu-?usp=sharing}

\newcommand{\FolderFormat}{Group Number - LastName1\_FirstName1, LastName2\_FirstName2}
\newcommand{\FolderFormatExample}{Group 1 - Doe\_John, Smith\_Jane}

% Book's title and subtitle
\title{\Huge \textbf{\BookTitle}  \footnote{\BookTitleFootnote} \\
\huge \BookSubtitle \footnote{\BookSubtitleFootnote}}

% Author
\author{\textsc{\BookAuthorName}\thanks{\url{\BookAuthorURL}}}

\begin{document}

\frontmatter
\date{}
\maketitle

%%%%%%%%%%%%%%%%%%%%%%%%%%%%%%%%%%%%%%%%%%%%%%%%%%%%%%%%%%%%%%%
% Add a dedication paragraph to dedicate your book to someone %
%%%%%%%%%%%%%%%%%%%%%%%%%%%%%%%%%%%%%%%%%%%%%%%%%%%%%%%%%%%%%%%
\begin{dedication}
Sorsogon State University - Bulan Campus
\end{dedication}

%%%%%%%%%%%%%%%%%%%%%%%%%%%%%%%%%%%%%%%%%%%%%%%%%%%%%%%%%%%%%%%%%%%%%%%%
% Auto-generated table of contents, list of figures and list of tables %
%%%%%%%%%%%%%%%%%%%%%%%%%%%%%%%%%%%%%%%%%%%%%%%%%%%%%%%%%%%%%%%%%%%%%%%%

\mainmatter

%%%%%%%%%%%
% Preface %
%%%%%%%%%%%
\section*{Coding Exercises}

Instructions: Write a program that solves the following problems.
Submit your code to the Google Drive folder provided by the instructor.

The following exercises are designed to help you practice using methods in Java.
You can create separate methods for each exercise and call them from the ``main''
method to test your solutions.

% Calculate the speed of a vehicle
\subsection*{Exercise 1: Calculate Speed}

In this exercise, you will write a Java program to calculate the speed of a
vehicle. The speed of a vehicle is calculated using the formula:

\begin{align}
  \text{Speed} &= \frac{\text{Distance}}{\text{Time}} \\
  \text{Distance} &= \text{Final Distance} - \text{Initial Distance} \\
  \text{Time} &= \text{Final Time} - \text{Initial Time} \\
  \text{Speed} &= \frac{\text{Final Distance} - \text{Initial Distance}}{\text{Final Time} - \text{Initial Time}}
\end{align}

\begin{enumerate}
  \item Create a method called ``calculateSpeed'' that accepts the initial
  distance, final distance, initial time, and final time as parameters and
  returns the speed of the vehicle.
  \item Calculate the speed of a vehicle that travels from an initial distance
  of 0 meters to a final distance of 100 meters in 10 seconds.
  \item Print the speed of the vehicle to the console.
  \begin{quote}
    Example: \\
    A vehicle left from 0 meters and traveled to 100 meters in 10 seconds.
    The speed of the vehicle is 10 meters per second.

    $Initial Distance: 0$ \\
    $Final Distance: 100$ \\
    $Initial Time: 0$ \\
    $Final Time: 10$ \\

    $Speed = \frac{100 - 0}{10 - 0} = \frac{100}{10} = 10$ \\
    $Speed = 10$ meters per second

    \textbf{Output:} \\
    The vehicle traveled from 0 meters to 100 meters in 0 seconds to 10 seconds
    at a speed of 10 meters per second.
  \end{quote}
\end{enumerate}

% Initial Code for Exercise 1
\begin{lstlisting}[language=Java, caption={Initial Code for Exercise 1}, label={lst:ce3-ex1}]
  // CalculateSpeed.java
  package com.oop.Exercises;

  import java.util.Scanner;

  public class CalculateSpeed {
      public static void main(String[] args) {
          Scanner scanner = new Scanner(System.in);

          double finalDistance = 0, initialDistance = 0, finalTime = 0, initialTime = 0;

          do {
              if (finalDistance <= initialDistance) {
                  System.out.println("\nFinal distance should be greater than initial distance.");
              }

              System.out.print("Enter the initial distance: ");
              initialDistance = scanner.nextDouble();
              System.out.print("Enter the final distance: ");
              finalDistance = scanner.nextDouble();
          } while (finalDistance <= initialDistance);

          do {
              if (finalTime <= initialTime) {
                  System.out.println("\ndFinal time should be greater than initial time.");
              }

              System.out.print("Enter the initial time: ");
              initialTime = scanner.nextDouble();
              System.out.print("Enter the final time: ");
              finalTime = scanner.nextDouble();
          } while (finalTime <= initialTime);

          double speed = calculateSpeed(initialDistance, finalDistance, initialTime, finalTime);

          String output = String.format(
                  "The vehicle traveled from %.2f meters to %.2f meters in %.2f seconds to %.2f seconds at a speed of %.2f meters per second.",
                  initialDistance, finalDistance, initialTime, finalTime, speed);

          System.out.println("\n" + output);

          scanner.close();
      }

      public static double calculateSpeed(double initialDistance, double finalDistance, double initialTime,
              double finalTime) {
          // Calculate the speed of the vehicle
      }
  }
\end{lstlisting}

Complete the code in Code \ref{lst:ce3-ex1} by implementing the ``calculateSpeed''
method to calculate the speed of the vehicle. Test the program by entering the
initial distance, final distance, initial time, and final time from the user and
printing the speed of the vehicle to the console.

% Arithmetic Sequence using Recursion
\subsection*{Exercise 2: Arithmetic Sequence}

In this exercise, you will write a Java program to calculate the sum of an
arithmetic sequence using recursion. An arithmetic sequence is a sequence of
numbers in which the difference between consecutive terms is constant. The sum
of an arithmetic sequence is calculated using the formula:

\begin{align}
  \text{Sum} = \frac{n}{2} \times (2a + (n - 1)d)
\end{align}

Where:
\begin{itemize}
  \item $n$ is the number of terms in the sequence.
  \item $a$ is the first term in the sequence.
  \item $d$ is the common difference between consecutive terms.
  \item $2a + (n - 1)d$ is the sum of the first and last terms in the sequence.
  \item $\frac{n}{2} \times (2a + (n - 1)d)$ is the sum of the arithmetic sequence.
\end{itemize}

\begin{enumerate}
  \item Create a method called ``calculateArithmeticSequence'' that accepts the
  first term, common difference, number of terms, and current term as parameters
  and returns the sum of the arithmetic sequence.
  \item Calculate the sum of an arithmetic sequence with the first term 1, common
  difference 2, and number of terms 5.
  \begin{quote}
    Base Case: If the current term is equal to the number of terms, return the
    current term. \\
    Recursive Case: Calculate the sum of the arithmetic sequence using recursion.
  \end{quote}
  \item Print the sum of the arithmetic sequence to the console.
  \begin{quote}
    Example: \\
    If the first term is 1, the common difference is 2, and the number of terms
    is 5, then the sum of the arithmetic sequence is 25. \\

    $Sequence: 1, 3, 5, 7, 9$ \\
    $Sum: 1 + 3 + 5 + 7 + 9 = 25$
  \end{quote}
\end{enumerate}

% Initial Code for Exercise 2
\begin{lstlisting}[language=Java, caption={Initial Code for Exercise 2}, label={lst:ce3-ex2}]
  // ArithmeticSequence.java
  package com.oop.Exercises;

  import java.util.Scanner;

  public class ArithmeticSequence {
      public static void main(String[] args) {
          Scanner scanner = new Scanner(System.in);

          System.out.print("Enter the first term of the arithmetic sequence: ");
          double firstTerm = scanner.nextDouble();
          System.out.print("Enter the common difference of the arithmetic sequence: ");
          double commonDifference = scanner.nextDouble();
          System.out.print("Enter the number of terms in the arithmetic sequence: ");
          int numberOfTerms = scanner.nextInt();

          double sum = calculateArithmeticSequence(firstTerm, commonDifference, numberOfTerms, 1);

          String output = String.format(
                  "The sum of the arithmetic sequence with the first term %.2f, common difference %.2f, and %d terms is %.2f.",
                  firstTerm, commonDifference, numberOfTerms, sum);

          System.out.println("\n" + output);

          scanner.close();
      }

      public static double calculateArithmeticSequence(double firstTerm, double commonDifference, int numberOfTerms,
              int currentTerm) {
          // Calculate the sum of the arithmetic sequence using recursion
      }
  }
\end{lstlisting}

Complete the code in Code \ref{lst:ce3-ex2} by implementing the
``calculateArithmeticSequence'' method to calculate the sum of the arithmetic
sequence using recursion. Test the program by entering the first term, common
difference, and number of terms from the user and printing the sum of the
arithmetic sequence to the console.

\section*{Submission of Coding Exercises}

Instructions:
\begin{enumerate}
  \item Go to the Google Drive folder provided by the instructor: \\
  \begin{quote}
    \textbf{For BSCS 2-1:} \\
    \url{\GoogleDriveURLBSCSTwoOne} \\ \\
    \textbf{For BSCS 2-2:} \\
    \url{\GoogleDriveURLBSCSTwoTwo}
  \end{quote}
  \item Inside the folder, create another folder for your group
  with the following format:
    \begin{quote}
      \textbf{\FolderFormat} \\
      Example: \textbf{\FolderFormatExample}
    \end{quote}
  \item Inside the sub-folder, create another folder with the name:
    \begin{quote}
        \textbf{Chapter \ChapterNumber - Coding Exercise \CodingNumber - \CodingTitle}
    \end{quote}
  \item Inside the folder, upload the file of your submission.
    \begin{quote}
        Fill in the template provided in the following link and upload
        it inside the folder: \\
        \url{\SubmissionTemplateURL}
    \end{quote}
  \item The activity must be submitted \textbf{\SubmissionDeadlineText}.
  \item Late submissions will not be accepted.
\end{enumerate}

\end{document}