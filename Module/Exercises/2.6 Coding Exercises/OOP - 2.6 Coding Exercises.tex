%%%%%%%%%%%%%%%%%%%%%%%%%%%%%%%%%%%%%%%%%%%%%%%%%%%
%% LaTeX book template                           %%
%% Author:  Amber Jain (http://amberj.devio.us/) %%
%% License: ISC license                          %%
%%%%%%%%%%%%%%%%%%%%%%%%%%%%%%%%%%%%%%%%%%%%%%%%%%%

\documentclass[a4paper,11pt,oneside]{book}
\usepackage{../../modulestyle}

%%%%%%%%%%%%%%%%%%%%%%%%%%%%%%%%%%%%%%%%%%%%%%%%%%%%%%%%%
% Source: http://en.wikibooks.org/wiki/LaTeX/Hyperlinks %
%%%%%%%%%%%%%%%%%%%%%%%%%%%%%%%%%%%%%%%%%%%%%%%%%%%%%%%%%

%%%%%%%%%%%%%%%%%%%%%%%%%%%%%%%%%%%%%%%%%%%%%%%%%%%%%%%%%%%%%%%%%%%%%%%%%%%%%%%%
% 'dedication' environment: To add a dedication paragraph at the start of book %
% Source: http://www.tug.org/pipermail/texhax/2010-June/015184.html            %
%%%%%%%%%%%%%%%%%%%%%%%%%%%%%%%%%%%%%%%%%%%%%%%%%%%%%%%%%%%%%%%%%%%%%%%%%%%%%%%%
\newenvironment{dedication}
{
   \cleardoublepage
   \thispagestyle{empty}
   \vspace*{\stretch{1}}
   \hfill\begin{minipage}[t]{0.66\textwidth}
   \raggedright
}
{
   \end{minipage}
   \vspace*{\stretch{3}}
   \clearpage
}

%%%%%%%%%%%%%%%%%%%%%%%%%%%%%%%%%%%%%%%%%%%%%%%%
% Chapter quote at the start of chapter        %
% Source: http://tex.stackexchange.com/a/53380 %
%%%%%%%%%%%%%%%%%%%%%%%%%%%%%%%%%%%%%%%%%%%%%%%%
\makeatletter
\renewcommand{\@chapapp}{}% Not necessary...
\newenvironment{chapquote}[2][2em]
  {\setlength{\@tempdima}{#1}%
   \def\chapquote@author{#2}%
   \parshape 1 \@tempdima \dimexpr\textwidth-2\@tempdima\relax%
   \itshape}
  {\par\normalfont\hfill--\ \chapquote@author\hspace*{\@tempdima}\par\bigskip}
\makeatother

%%%%%%%%%%%%%%%%%%%%%%%%%%%%%%%%%%%%%%%%%%%%%%%%%%%
% First page of book which contains 'stuff' like: %
%  - Book title, subtitle                         %
%  - Book author name                             %
%%%%%%%%%%%%%%%%%%%%%%%%%%%%%%%%%%%%%%%%%%%%%%%%%%%

\newcommand{\CourseTitle}{Object-Oriented Programming}
\newcommand{\ChapterNumber}{2}
\newcommand{\ChapterTitle}{Control Statements}
\newcommand{\CodingNumber}{2.6}
\newcommand{\CodingTitle}{Control Flow: The Road Less Traveled}
\newcommand{\SubmissionDeadline}{October 11, 2024}
\newcommand{\SubmissionDeadlineText}{on or before \SubmissionDeadline}
\newcommand{\SubmissionTemplateURL}{https://docs.google.com/document/d/1sctvVLgpPSVnXN82k6LsOPSPApKp2rV0/edit?usp=drive_link&ouid=112709378145681657270&rtpof=true&sd=true}

\newcommand{\BookTitle}{Coding Exercise: \CodingNumber - \CodingTitle}
\newcommand{\BookTitleFootnote}{A coding exercise for
Chapter \ChapterNumber of the Study Guide on the course \CourseTitle.}

\newcommand{\BookSubtitle}{Chapter \ChapterNumber: \ChapterTitle}
\newcommand{\BookSubtitleFootnote}{This chapter introduces the basic concepts
of control statements in Java programming language.}

\newcommand{\BookAuthorFirstName}{Jarrian Vince}
\newcommand{\BookAuthorLastName}{Gojar}
\newcommand{\BookAuthorName}{Jarrian Vince G. Gojar}
\newcommand{\BookAuthorURL}{https://github.com/godkingjay}

\newcommand{\GoogleDriveURLBSCSTwoOne}{https://drive.google.com/drive/folders/1c56xFCJgFh6FWQQ4iZ-UuKKcWioF8pgs?usp=sharing}
\newcommand{\GoogleDriveURLBSCSTwoTwo}{https://drive.google.com/drive/folders/1jANc3o6atOYbHyoJZ6b-j-nDlTknEiu-?usp=sharing}

\newcommand{\FolderFormat}{Group Number - LastName1\_FirstName1, LastName2\_FirstName2}
\newcommand{\FolderFormatExample}{Group 1 - Doe\_John, Smith\_Jane}

% Book's title and subtitle
\title{\Huge \textbf{\BookTitle}  \footnote{\BookTitleFootnote} \\
\huge \BookSubtitle \footnote{\BookSubtitleFootnote}}

% Author
\author{\textsc{\BookAuthorName}\thanks{\url{\BookAuthorURL}}}

\begin{document}

\frontmatter
\date{}
\maketitle

%%%%%%%%%%%%%%%%%%%%%%%%%%%%%%%%%%%%%%%%%%%%%%%%%%%%%%%%%%%%%%%
% Add a dedication paragraph to dedicate your book to someone %
%%%%%%%%%%%%%%%%%%%%%%%%%%%%%%%%%%%%%%%%%%%%%%%%%%%%%%%%%%%%%%%
\begin{dedication}
Sorsogon State University - Bulan Campus
\end{dedication}

%%%%%%%%%%%%%%%%%%%%%%%%%%%%%%%%%%%%%%%%%%%%%%%%%%%%%%%%%%%%%%%%%%%%%%%%
% Auto-generated table of contents, list of figures and list of tables %
%%%%%%%%%%%%%%%%%%%%%%%%%%%%%%%%%%%%%%%%%%%%%%%%%%%%%%%%%%%%%%%%%%%%%%%%

\mainmatter

%%%%%%%%%%%
% Preface %
%%%%%%%%%%%
\section*{The Scanner Class}

The ``Scanner'' class can be used to read input from the user. The following
exercises are designed to help you practice using control statements in Java.
You can use the ``Scanner'' class to read input from the user and test your
solutions.

% Simple Implementation of the Scanner class
\begin{lstlisting}[language=Java, caption={Simple Implementation of the Scanner Class}]
  // ScannerExample.java
  import java.util.Scanner;

  public class ScannerExample {
    public static void main(String[] args) {
      Scanner scanner = new Scanner(System.in);

      System.out.print("Enter a number: ");
      int number = scanner.nextInt();
      System.out.println("You entered: " + number);

      scanner.close();
    }
  }
\end{lstlisting}

In the above code, we import the ``Scanner'' class by using the statement
``import java.util.Scanner;''. We create an instance of the ``Scanner''
class by using the statement ``Scanner scanner = new Scanner(System.in);''.
We read an integer input from the user by using the statement
``int number = scanner.nextInt();''. We print the input number by using
the statement ``System.out.println("You entered: " + number);''. Finally,
we close the scanner by using the statement ``scanner.close();''.

The following are used to read different types of input from the user:
\begin{itemize}
  \item ``scanner.nextBoolean()'' to read a boolean value.
  \item ``scanner.nextByte()'' to read a byte value.
  \item ``scanner.nextShort()'' to read a short value.
  \item ``scanner.nextInt()'' to read an integer value.
  \item ``scanner.nextLong()'' to read a long value.
  \item ``scanner.nextFloat()'' to read a float value.
  \item ``scanner.nextDouble()'' to read a double value.
  \item ``scanner.nextLine()'' to read a string value.
  \item ``scanner.next()'' to read a single word.
\end{itemize}

\section*{Coding Exercises}

Instructions: Write a program that solves the following problems.
Submit your code to the Google Drive folder provided by the instructor.

% Red Large Text
\textcolor{red}{\Large{Note 1: You have to answer 3 out of the 5 exercises.}}

% Answer 1 and one of the remaining exercises
\textcolor{red}{\Large{Note 2: You have to answer Exercise 1, Exercise 5, and
one of the remaining exercises. (e.g., Exercise 2, Exercise 3, Exercise 4) for
a total of 3 exercises.}}

\textcolor{red}{\Large{Note 3: In the following exercises, you will use the
``Scanner'' class to read input from the user.}}

\subsection*{Exercise 1: Find the Largest Number (Conditional Statements)}

In this exercise, you will write a Java program to find the largest of three
numbers entered by the user. You should use the ``Scanner'' class to read
input from the user and conditional statements to compare the three numbers
and determine the largest number. Finally, you should print the largest
number to the console. If the numbers are equal, you should print a message
saying "There is no largest number".

\begin{enumerate}
  \item With the use of the ``Scanner'' class, read three numbers from the user.
  and store them in variables ``num1'', ``num2'', and ``num3''.
  \item Use conditional statements to compare the three numbers and determine
  the largest number.
  \begin{enumerate}
    \item If ``num1'' is greater than ``num2'' and ``num1'' is greater than ``num3'',
    then ``num1'' is the largest number and print "The largest number is: {num1}".
    \begin{quote}
      Example:
      If ``num1'' is 5, ``num2'' is 3, and ``num3'' is 4, then the output should
      be "The largest number is: 5".
    \end{quote}
    \item If none of the above conditions are true, then the numbers are equal. Thus,
    print a message saying "There is no largest number".
    \begin{quote}
      Example:
      If ``num1'' is 5, ``num2'' is 5, and ``num3'' is 5, then the output should
      be "There is no largest number".
    \end{quote}
  \end{enumerate}
\end{enumerate}

\subsection*{Exercise 2: Print Multiplication Table (Iteration Statements)}

In this exercise, you will write a Java program to print the multiplication
table of a number entered by the user. You should use the ``Scanner'' class
to read input from the user and iteration statements to print the multiplication
table of the entered number. Finally, you should print the multiplication table
to the console.

\begin{enumerate}
  \item With the use of the ``Scanner'' class, read a number from the user and
  store it in a variable ``number''.
  \item Use a ``for'' loop to iterate from 1 to 10.
  \begin{enumerate}
    \item Inside the loop, calculate the product of ``number'' and the loop
    variable and store it in a variable ``result''.
    \item Print the multiplication table in the format "{number} * {loop variable}
    = {result}".
    \begin{quote}
      Example: \\
      If the user enters 5, then the output should be: \\
      5 * 1 = 5 \\
      5 * 2 = 10 \\
      5 * 3 = 15 \\
      5 * 4 = 20 \\
      5 * 5 = 25 \\
      5 * 6 = 30 \\
      5 * 7 = 35 \\
      5 * 8 = 40 \\
      5 * 9 = 45 \\
      5 * 10 = 50
    \end{quote}
  \end{enumerate}
\end{enumerate}

\subsection*{Exercise 3: Factorial of a Number (Iteration Statements)}

In this exercise, you will write a Java program to calculate the factorial of
a number entered by the user. You should use the ``Scanner'' class to read input
from the user and iteration statements to calculate the factorial of the entered
number. Finally, you should print the factorial to the console.

A factorial of a non-negative integer ``n'' is the product of all positive
integers less than or equal to ``n''. It is denoted by ``n!''.

\begin{align}
  n! = n \times (n - 1) \times (n - 2) \times \ldots \times 3 \times 2 \times 1
\end{align}

The formula above shows how to calculate the factorial of a number.

Example:
The factorial of 5 is calculated as follows:

\begin{align}
  5! = 5 \times 4 \times 3 \times 2 \times 1 = 120
\end{align}

In this exercise, you will calculate the factorial of a number entered by the
user.

\begin{enumerate}
  \item With the use of the ``Scanner'' class, read a number from the user and
  store it in a variable ``number''.
  \item Initialize a variable ``factorial'' to 1.
  \item Use a ``for'' loop to iterate from 1 to ``number''.
  \begin{enumerate}
    \item Inside the loop, multiply the loop variable by ``factorial'' and store
    the result in ``factorial''.
    \item Print the factorial of the entered number.
    \begin{quote}
      Example: \\
      If the user enters 5, then the output should be: \\

      \begin{align}
        5! = 5 \times 4 \times 3 \times 2 \times 1 = 120
      \end{align}

      Thus, the factorial of 5 is 120.
    \end{quote}
  \end{enumerate}
\end{enumerate}

\subsection*{Exercise 4: Fibonacci Series (Iteration Statements)}

In this exercise, you will write a Java program to print the Fibonacci series
up to a specified number of terms entered by the user. You should use the
``Scanner'' class to read input from the user and iteration statements to
calculate and print the Fibonacci series. Finally, you should print the
Fibonacci series to the console.

The Fibonacci series is a series of numbers in which each number is the sum
of the two preceding ones, usually starting with 0 and 1. The sequence goes
0, 1, 1, 2, 3, 5, 8, 13, 21, and so on.

\begin{align}
  F_0 = 0, F_1 = 1, F_n = F_{n-1} + F_{n-2} \text{ for } n > 1
\end{align}

The formula above shows how to calculate the Fibonacci series.

Example:
The Fibonacci series up to 10 terms is calculated as follows:

\begin{align}
  0, 1, 1, 2, 3, 5, 8, 13, 21, 34
\end{align}

In this exercise, you will calculate and print the Fibonacci series up to a
specified number of terms entered by the user.

\begin{enumerate}
  \item With the use of the ``Scanner'' class, read the number of terms from
  the user and store it in a variable ``terms''.
  \item Initialize two variables ``firstTerm'' and ``secondTerm'' to 0 and 1
  respectively.
  \item Print the first two terms of the Fibonacci series.
  \item Use a ``for'' loop to iterate from 1 to ``terms''.
  \begin{enumerate}
    \item Inside the loop, calculate the next term of the Fibonacci series by
    adding ``firstTerm'' and ``secondTerm'' and store it in a variable ``nextTerm''.
    \item Print the next term of the Fibonacci series.
    \item Update the values of ``firstTerm'' and ``secondTerm'' to the previous
    two terms of the Fibonacci series.
    \begin{quote}
      Example: \\
      If the user enters 10, then the output should be: \\

      \begin{align}
        0, 1, 1, 2, 3, 5, 8, 13, 21, 34
      \end{align}

      Thus, the Fibonacci series up to 10 terms is 0, 1, 1, 2, 3, 5, 8, 13, 21, 34.
    \end{quote}
  \end{enumerate}
\end{enumerate}

% Jump Statement Exercises
\subsection*{Exercise 5: Sum of Numbers (Jump Statements)}

In this exercise, you will write a Java program to calculate the sum of numbers
entered by the user. You should use the ``Scanner'' class to read input from the
user and jump statements to control the flow of the program. Finally, you should
print the sum of the numbers to the console.

\begin{enumerate}
  \item With the use of the ``Scanner'' class, read numbers from the user until
  the user enters a negative number. Store the numbers in a variable ``number''
  and the sum of the numbers in a variable ``sum''.
  \item Use a ``while'' loop to read numbers from the user until the user enters
  a negative number.
  \begin{enumerate}
    \item Inside the loop, check if the value of ``number'' is negative. If it is,
    break out of the loop.
    \item Add the value of ``number'' to the sum of the numbers.
    \item Continue reading numbers from the user.
    \begin{quote}
      Example: \\
      If the user enters 5, 10, 15, -1, then the output should be: \\
      The sum of the numbers is: 30

      If the user enters 10, 20, 30, 40, 50, -1, then the output should be: \\
      The sum of the numbers is: 150
    \end{quote}
  \end{enumerate}
\end{enumerate}

\section*{Submission of Coding Exercises}

Instructions:
\begin{enumerate}
  \item Go to the Google Drive folder provided by the instructor: \\
  \begin{quote}
    \textbf{For BSCS 2-1:} \\
    \url{\GoogleDriveURLBSCSTwoOne} \\ \\
    \textbf{For BSCS 2-2:} \\
    \url{\GoogleDriveURLBSCSTwoTwo}
  \end{quote}
  \item Inside the folder, create another folder for your group
  with the following format:
    \begin{quote}
      \textbf{\FolderFormat} \\
      Example: \textbf{\FolderFormatExample}
    \end{quote}
  \item Inside the sub-folder, create another folder with the name:
    \begin{quote}
        \textbf{Chapter \ChapterNumber - Coding Exercise \CodingNumber - \CodingTitle}
    \end{quote}
  \item Inside the folder, upload the file of your submission.
    \begin{quote}
        Fill in the template provided in the following link and upload
        it inside the folder: \\
        \url{\SubmissionTemplateURL}
    \end{quote}
  \item The activity must be submitted \textbf{\SubmissionDeadlineText}.
  \item Late submissions will not be accepted.
\end{enumerate}

\end{document}